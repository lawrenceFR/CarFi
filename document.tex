\documentclass[a4paper,11pt]{article}
\usepackage[hidelinks]{hyperref}
\hypersetup{
		colorlinks = true,
    	linkcolor = blue,
    	citecolor = blue,
    	linktoc=all
}
\usepackage{graphicx}
\usepackage[utf8]{inputenc}
\usepackage[T1]{fontenc}
\usepackage[francais]{babel}
\usepackage{xcolor}
\usepackage{hyphenat}
\usepackage{datetime}
\usepackage{wrapfig}
\usepackage[top=3cm, bottom=3cm, left=2.5cm, right=2.5cm]{geometry}


\begin{document}

	\fontsize{13pt}{18pt}\selectfont
	\begin{figure}[htbp]
		\begin{minipage}[b]{0.5\linewidth}
			\centering
			\includegraphics[width=\linewidth]{pictures/worldline.jpg}
		\end{minipage}
		\hspace{3cm}
		\begin{minipage}[b]{0.25\linewidth}
			\centering
			\includegraphics[width=\linewidth]{pictures/telecom.png}
		\end{minipage}
	\end{figure}

	\begin{center}
	\section*{Télécom ParisTech \\
	Promotion 2017 \\
	Sylvain DASSIER
	}
	\end{center}


	\vspace{2cm}
	\begin{center}
		\huge{\textbf{RAPPORT DE STAGE}} \\
		\vspace{1cm}
		\textbf{%
		\textcolor{blue}{Étude de l'apport du protocol MPTCP dans l'optimisation du trafic}} \\
		
	\end{center}



	\vspace{1cm}
	
		\textbf{
			\begin{tabbing}
		  		Département : ~~~~~~~~\= \textit{Département d'Informatique} \\
		 	 	Option : \> \textit{INFRES} \\
		  		Encadrants : \> \textit{M. Luigi IANNONE, M. Antoine FRESSANCOURT} \\
		  		Dates : \> \textit{18/07/2016 - 17/01/2017} \\
		 		Adresse : \> \textit{Télécom ParisTech, 23 Avenue d'Italie,} \\
		  				\> \textit{75013 Paris}
			\end{tabbing}
		}
		
	\clearpage
	
	\begin{center}
		\Huge{\textbf{Declaration d'intégrité relative au plagiat}}
	\end{center}
	\vspace{3cm}
	\emph{Je soussigné} DASSIER Sylvain \emph{certifie sur l'honneur :}\\
	\begin{enumerate}
	  	\item Que les résultats décrits dans ce rapport sont l'aboutissement de mon
	  	travail.
		\item Que je suis l'auteur de ce rapport.
		\item Que je n'ai pas utilisé des sources ou résultats tiers sans
		clairement les citer et les référencer selon les règles bibliographiques
		préconisées.
	\end{enumerate}
	\vspace{2cm}
	\includegraphics[scale=0.3]{pictures/declaration.jpg}
	
	\vspace{-1cm}
	\formatdate{30}{6}{2016}
	\hspace{5cm}
	Signature : \includegraphics[scale=0.5]{pictures/signa.jpg}
	
	\clearpage


	\section*{\Huge \textcolor{blue}{\textit{Abstract}}}
	
		\begin{description}
		
			\item \hspace{2cm} English
			
		\end{description}
		
	\section*{\Huge \textcolor{blue}{\textit{Résumé}}}
	
		\begin{description}

			\item \hspace{2cm} Français
			
		\end{description}
		
	

	\clearpage

	\setcounter{tocdepth}{3}

	\tableofcontents

	\clearpage

	
	\section{Introduction}

		\vspace{0.5cm}
		\subsection{Context}
			\begin{description}
				\item \hspace{2cm} Introduction
			\end{description}
	
		\vspace{0.5cm}
		\subsection{Document Outline}
			\begin{description}
				\item \hspace{2cm} In section \hyperref[sec:mptcplinuximplement]{2} we have described how to set up \textbf{\emph{The MPTCP Linux kernel implementation}}. In section \hyperref[sec:packetpath]{3} we trace the path taken by a packet during its journey using the protocol \textbf{\emph{MPTCP}}.
			\end{description}


	\clearpage
	\section{The \emph{$\mathbf{MPTCP}$} Linux kernel library compilation : }

		\label{sec:mptcplinuximplement}
		\hspace{2cm} The following figure illustrates the


		\subsection{Faliures :}

			\label{subsec:faliures}

			\hspace{2cm} The following section describes our attempts to put in place a debugging system for \emph{$\mathbf{MPTCP}$} so that we are not required to copile a kernel version everytime, which can take quite long.
			
			\begin{enumerate}

				\item LibOS with NUSE
				\item LibOS with DCE

				\subsubsection{LibOS with NUSE}
					\label{subsubsec:liboswithnuse}

				\subsubsection{LibOS with DCE}
					\label{subsubsec:liboswithdce}

					LibOS with DCE is put in place in the following manner :

					\begin{enumerate}

					\item \textbf{Install the dependencies :}
						
					\nohyphens{
						sudo apt-get install vim git mercurial gcc g++ python python-dev qt4-dev-tools libqt4-dev bzr cmake libc6-dev libc6-dev-i386 g++-multilib gdb valgrind gsl-bin libgsl0-dev libgsl0ldbl flex bison libfl-dev tcpdump sqlite sqlite3 libsqlite3-dev libxml2 libxml2-dev libgtk2.0-0 libgtk2.0-dev vtun lxc uncrustify doxygen graphviz imagemagick texlive texlive-extra-utils texlive-latex-extra texlive-font-utils dvipng python-sphinx dia python-pygraphviz python-kiwi python-pygoocanvas libgoocanvas-dev ipython libboost-signals-dev libboost-filesystem-dev openmpi-bin openmpi-common openmpi-doc libopenmpi-dev libncurses5-dev libncursesw5-dev unrar unrar-free p7zip-full autoconf libpcap-dev cvs libssl-dev wireshark}

					\item \textbf{Build DCE using bake :}

						\begin{enumerate}

							\item hg clone http://code.nsnam.org/bake bake
							\item export BAKE\_HOME=`pwd`/bake
							\item export PATH=\$PATH:\$BAKE\_HOME
							\item export PYTHONPATH=\$PYTHONPATH:\$BAKE\_HOME
							\item mkdir dce
							\item cd dce
							\item bake.py configure -e dce-ns3-1.8
							\item bake.py download
							\item bake.py build

						\end{enumerate}

					\item \textbf{Build the \emph{mptcp\_trunk\_libos} branch of \emph{net-next-nuse}}
						\begin{enumerate}

							\item git clone -b mptcp\_trunk\_libos https://github.com/libos-nuse/net-next-nuse.git
							\item cd net-next-nuse
							\item make menuconfig ARCH=lib
							\item make library ARCH=lib

						\end{enumerate}

					\item \textbf{Build \emph{iproute2} version \emph{2.6.38}}

						\begin{enumerate}
							\item Download the compressed source code from \\
							\nohyphens{\emph{https://kernel.googlesource.com/pub/scm/linux/kernel/git/shemminger/iproute2/+archive/fcae78992cab7bd267785b392b438306c621e583.tar.gz}} , extract it and rename the folder to \emph{iproute2-2.6.38}. 
							\item cd iproute2-2.6.38
							\item patch -p1 -i ../ns-3-dce/utils/iproute-2.6.38-fix-01.patch
							\item \$(KERNEL\_INCLUDE) should point to the liblinux.so directory ( for me it is \$HOME/net-next-nuse ) \\
    						Hence I modified the following part in the Makefile:

    						\emph{
    						Config: \\
                               \hspace*{2cm}sh configure /home/lawrence/net-next-nuse \\
                               \hspace*{2cm}\# sh configure \$(KERNEL\_MODULE) }
                             \item \nohyphens{LDFLAGS=-pie make CCOPTS='-fpic -D\_GNU\_SOURCE -O0 -U\_FORTIFY\_SOURCE'}

						\end{enumerate}

					\item \textbf{Set the \emph{DCE\_PATH}} \\
						export DCE\_PATH=\$HOME/net-next-nuse:\$HOME/iproute2-2.6.38/ip

					\item \textbf{Build \emph{ns-3-dce } with }
						\begin{enumerate}
						
							\item hg clone http://code.nsnam.org/ns-3-dce  -r dce-1.8
							\item cd ns-3-dce
							\item ./waf configure --with-ns3=\$HOME/dce/build --enable-kernel-stack=\$HOME/net-next-nuse/arch --prefix=\$HOME/dce/build
							\item ./waf build
							\item ./waf --run dce-iperf-mptcp

						\end{enumerate}

					\end{enumerate}

			\end{enumerate}






	\clearpage
	\section{The \emph{$\mathbf{MPTCP}$} Linux kernel implementation setup : }

		\label{sec:mptcplinuximplement}
		\hspace{2cm} The following figure illustrates the 

		
	\clearpage
	\section{Packet path using \emph{$\mathbf{MPTCP}$} :}

		\label{sec:packetpath}
		\vspace{0.5cm}
		\begin{description}
		\item \hspace{2cm} a
		\end{description}
  		
  		\subsection{Hypothesis}

  		 \begin{description}

  		 	\label{subsec:hypothesis}
  			\item \hspace{2cm} a
  			\vspace{1cm}
		\end{description}
			
		\subsection{Procedure}
			


		\section{Results}

			\label{sec:results}
		
			\hspace{2cm} This section 
			
			
			
		\clearpage
		\section{Conclusion}
		 	\begin{description}
		 	\item \hspace{2cm} Conclusion
		 	\end{description}
		 	
		\section{Further developements}
		\begin{description}
			\item \hspace{2cm} In the above experiments 
		\end{description}
		 	
		 	
		 \clearpage
		 \section{Acknowledgements}
		 
		  		Acknowledgement
		 
		 	\clearpage
		 	\section{Bibliography}
		\bibliographystyle{unsrt}
		\bibliography{library}
		\clearpage
  		\section{Appendix}
  		 	Here we have the different 
		\section{Glossary}
				\begin{tabbing}
		  		RA : ~~~~~~~~\= \textit{Département d'Informatique} \\
		 	 	IETF : \> \textit{Internet Engineering Task Force} \\
		  		L2 : \> \textit{Layer 2/Link Layer of the OSI model} \\
		  		L3 : \> \textit{Layer 2/IP Layer of the OSI model} \\
		 		DHCP : \> \textit{Dynamic Host Configuration Protocol} \\
		  		DNS : \> \textit{Domain name system} \\
		  		MLD : \> \textit{Multicast Listener Discovery} \\
		  		IP : \> \textit{Internet Protocol} \\
		  		RFC : \> \textit{Request for Comment} \\
		  		ARP : \> \textit{Address Resolution Protocol} \\
		  		VLAN : \> \textit{Virtual local area network} \\
		  		AP : \> \textit{Access Point} \\
		  		RS : \> \textit{Router Solicitation} \\
		  		NS : \> \textit{Neighbour Solicitation} \\
		  		NA : \> \textit{Neighbour Advertisement} \\
		  		mDNS : \> \textit{ multicast Domain Name System} \\
		  		LLMNR : \> \textit{ Link-Local Multicast Name Resolution} \\
		  		SLAAC : \> \textit{ Stateless Address Autoconfiguration} \\
				\end{tabbing}
				
				
				

\end{document}